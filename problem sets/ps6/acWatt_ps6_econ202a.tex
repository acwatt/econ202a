\documentclass[12pt]{article}
\usepackage{{../preamble}} % for use when the .tex file is in a psX subfolder
\graphicspath{{pics}}
\begin{document}
% \maketitle
\chead{Problem Set 1}

%%%%%%%%%%%%%%%%%%%%%%%%%%%%%%%%%%%%%%%%%%%%%%%
%                  Definitions                %
%%%%%%%%%%%%%%%%%%%%%%%%%%%%%%%%%%%%%%%%%%%%%%%
%\includegraphics[width=\mywidth\textwidth]{}

% \begin{figure}[h!]
% \centering
% \input{pics/PS2/p7b}
% \caption{}
% \label{fig-}
% \end{figure}

% \begin{enumerate}[label=\alph*.]
%     \setcounter{enumi}{1}
%     \item 
% \end{enumerate}

%%%%%%%%%%%%%%%%%
%     Part a    %
%%%%%%%%%%%%%%%%%


%%%%%%%%%%%%%%%%%%%%%%%%%%%%%%%%%%%%%%%%%%%%%%%
%                Problem 1                    %
%%%%%%%%%%%%%%%%%%%%%%%%%%%%%%%%%%%%%%%%%%%%%%%
\section*{Problem 1}
\problem{Eat-the-Pie Problem}{
    Consider a household that must live forever off of an initial stock of wealth $A_0$ that pays a return $R$. The household seeks to maximize the utility function
    \[\sum\limits_{t=0}^\infty \beta^t u(C_t)\]
    The household's wealth evolves according to 
    \[A_{t+1}=R(A_t-C_t)\]
    The Bellman equation for the household's problem is
    \[V(A)=\max_{C\in[0,A]}\l\{ u(C)+\beta V(R(A-C)) \r\}\]
}

\begin{align} 
    \intertext{stuff}
    y&=x
\end{align}


%%%%%%%%%%%%%%%%%%%%%%%%%%%%%%%%%%%%%%%%%%%%%%%
%                Problem 2                    %
%%%%%%%%%%%%%%%%%%%%%%%%%%%%%%%%%%%%%%%%%%%%%%%
\newpage
\section*{Problem 2}
\problem{Problem name}{
    Problem description
}
\begin{align} 
    \intertext{stuff}
    y&=x
\end{align}



%%%%%%%%%%%%%%%%%%%%%%%%%%%%%%%%%%%%%%%%%%%%%%%
%                Problem 3                    %
%%%%%%%%%%%%%%%%%%%%%%%%%%%%%%%%%%%%%%%%%%%%%%%
\newpage
\section*{Problem 3}
\problem{Problem name}{
    Problem description
}

\begin{align} 
    \intertext{stuff}
    y&=x
\end{align}


%%%%%%%%%%%%%%%%%%%%%%%%%%%%%%%%%%%%%%%%%%%%%%%
%                Problem 4                    %
%%%%%%%%%%%%%%%%%%%%%%%%%%%%%%%%%%%%%%%%%%%%%%%
\newpage
\section*{Problem 4}
\problem{Problem name}{
    Problem description
}

\begin{align} 
    \intertext{stuff}
    y&=x
\end{align}



%%%%%%%%%%%%%%%%%%%%%%%%%%%%%%%%%%%%%%%%%%%%%%%
%                Problem 5                    %
%%%%%%%%%%%%%%%%%%%%%%%%%%%%%%%%%%%%%%%%%%%%%%%
\newpage
\section*{Problem 5}
\problem{Problem name}{
    Problem description
}

\begin{align} 
    \intertext{stuff}
    y&=x
\end{align}


%%%%%%%%%%%%%%%%%%%%%%%%%%%%%%%%%%%%%%%%%%%%%%%
%                Problem 6                    %
%%%%%%%%%%%%%%%%%%%%%%%%%%%%%%%%%%%%%%%%%%%%%%%
\section*{Problem 6}
\problem{Problem name}{
    Problem description
}

\begin{align} 
    \intertext{stuff}
    y&=x
\end{align}



%%%%%%%%%%%%%%%%%%%%%%%%%%%%%%%%%%%%%%%%%%%%%%%
%                Problem 7                    %
%%%%%%%%%%%%%%%%%%%%%%%%%%%%%%%%%%%%%%%%%%%%%%%
\section*{Problem 7}
\problem{Problem name}{
    Problem description
}

\begin{align} 
    \intertext{stuff}
    y&=x
\end{align}


%%%%%%%%%%%%%%%%%%%%%%%%%%%%%%%%%%%%%%%%%%%%%%%
%                Problem 8                    %
%%%%%%%%%%%%%%%%%%%%%%%%%%%%%%%%%%%%%%%%%%%%%%%
\section*{Problem 8}
\problem{Problem name}{
    Problem description
}

\begin{align} 
    \intertext{stuff}
    y&=x
\end{align}



%%%%%%%%%%%%%%%%%%%%%%%%%%%%%%%%%%%%%%%%%%%%%%%
%                Problem 9                    %
%%%%%%%%%%%%%%%%%%%%%%%%%%%%%%%%%%%%%%%%%%%%%%%
\section*{Problem 9}
\problem{Problem name}{
    Problem description
}

\begin{align} 
    \intertext{stuff}
    y&=x
\end{align}


%%%%%%%%%%%%%%%%%%%%%%%%%%%%%%%%%%%%%%%%%%%%%%%
%                Problem 10                   %
%%%%%%%%%%%%%%%%%%%%%%%%%%%%%%%%%%%%%%%%%%%%%%%
\section*{Problem 10}
\problem{Problem name}{
    Problem description
}

\begin{align} 
    \intertext{stuff}
    y&=x
\end{align}




\end{document}

