\documentclass[12pt]{article}
\usepackage{{../preamble}}
\graphicspath{{pics}}

\begin{document}
\chead{Problem Set 2}

%%%%%%%%%%%%%%%%%%%%%%%%%%%%%%%%%%%%%%%%%%%%%%%
%                  Definitions                %
%%%%%%%%%%%%%%%%%%%%%%%%%%%%%%%%%%%%%%%%%%%%%%%
\def\xdot{\dot X} \def\ydot{\dot Y} \def\zdot{\dot Z}
\def\kdot{\dot k} \def\ccdot{\dot c} \def\ddt{\frac{d}{dt}}
\def\dfdk{\frac{\partial F}{\partial K}}
\def\dfdl{\frac{\partial F}{\partial L}}
\def\plim{\lim\limits_{\phi\to 0}}
\def\ddp{\dfrac{\partial}{\partial \phi}}
\def\dgdp{\dfrac{\partial\gamma}{\partial \phi}}
\def\invphi{\sfrac{1}{\phi}}




%%%%%%%%%%%%%%%%%%%%%%%%%%%%%%%%%%%%%%%%%%%%%%%
%                Problem 1                    %
%%%%%%%%%%%%%%%%%%%%%%%%%%%%%%%%%%%%%%%%%%%%%%%
\section*{Problem 1}
\problem{Growth, saving, and $r-g$: Romer 2.6}{
    Piketty (2014) argues that a fall in the growth rate
of the economy is likely to lead to an increase in the difference between the real
interest rate and the growth rate. This problem asks you to investigate this issue
in the context of the Ramsey Cass Koopmans model. Specifically, consider a
Ramsey Cass Koopmans economy that is on its balanced growth path, and suppose
there is a permanent fall in $g$.

    \begin{enumerate}[label=(\alph*)]
    \item How, if at all, does this affect the $\kdot = 0$ curve?
    \end{enumerate} 
}


\includegraphics[width=0.5\textwidth]{1.a}


In the plot above, we can see a decrease in $g$
\begin{itemize}
    \item[$\implies$] a decrease in the slope of $(n+g)k$
    \item[$\implies$] increase in the $k$ where $f(k)-(n+g)k=0$
    \item[$\implies$] increase in the $k$ where the $\kdot=0$ curve meets the $k$ axis
\end{itemize}

A decrease in the slope of $(n+g)k$ also implies that the distance between $(n+g)k$ and $f(k)$ will be larger $\forall k$ between 0 and where $f(k)-(n+g)k=0$.

\noindent
So, overall, a decrease in $g$ implies a vertical and horizontal stretch outward for the $\kdot = 0$ curve.


%%%%%%%%%%%%%%%%%
%     Part b    %
%%%%%%%%%%%%%%%%%
\newpage\problem{}{
    \begin{enumerate}[label=(\alph*)]
    \setcounter{enumi}{1}
    \item How, if at all, does this affect the $\ccdot = 0$ curve?
    \end{enumerate} 
}

\includegraphics[width=0.5\textwidth]{1.b}

Because $f'>0$ and $f''<0$ by assumption, the plot of $f'$ is shaped as in the above plot. Then when $g$ has decreased from $g_1$ to $g_2$, the intersection with $f'$ occurs at a higher $k^*$. Since $\ccdot=0$ occurs when $f'=\rho+\theta g$ and $\rho, \theta$ are constants, then the $\ccdot=0$ curve shifts to the right to a higher $k^*$.
  

%%%%%%%%%%%%%%%%%
%     Part c    %
%%%%%%%%%%%%%%%%%
\newpage\problem{}{
    \begin{enumerate}[label=(\alph*)]
    \setcounter{enumi}{2}
    \item At the time of the change, does $c$ rise, fall, or stay the same, or is it not possible to tell?
    \end{enumerate} 
}

Since $k_1^* < k_2^*$ from the plot in part (b)
\begin{itemize}
    \item[$\implies$] immediately after the decrease from $g_1$ to $g_2$, $f'(k_1^*)>\rho+\theta g_2$.
    \item[$\implies$] $\dfrac{\ccdot}{c}\bigg|_{k=k_1^*,g=g_2} = \dfrac{f'(k_1^*) - \rho - \theta g_2}{\theta} > 0$
    \item[$\implies$] $\ccdot|_{k=k_1^*,g=g_2} > 0$
    \item[$\implies$] $c$ rises immediately after the decrease in $g$
\end{itemize}


%%%%%%%%%%%%%%%%%
%     Part d    %
%%%%%%%%%%%%%%%%%
\newpage\problem{}{
    \begin{enumerate}[label=(\alph*)]
    \setcounter{enumi}{3}
    \item At the time of the change, does $r-g$ rise, fall, or stay the same, or is it not possible to tell?
    \end{enumerate} 
}

Since there is no depreciation, the real interest rate $r$ is equal to the marginal product of capital $f'(k)$ (since the price of output is assumed to be 1).

\begin{align*}
    \intertext{So}
    r-g &= f'(k) - g
    \intertext{And immediately after the change in g,}
    f'(k_1^*) - g_1
    \intertext{changes to}
    f'(k_1^*) - g_2
    \intertext{Since $f'(k)$ remains unchanged at the time of the change but $g$ decreases from $g_1$ to $g_2$,}
    f'(k_1^*) - g_2 &>  f'(k_1^*) - g_1\\
    r|_{g_1}-g_2 &> r|_{g_1}-g_1
    \intertext{So, at the time of the change, $r-g$ increases.}
\end{align*}



%%%%%%%%%%%%%%%%%
%     Part e    %
%%%%%%%%%%%%%%%%%
\newpage\problem{}{
    \begin{enumerate}[label=(\alph*)]
    \setcounter{enumi}{4}
    \item In the long run, does $r-g$ rise, fall, or stay the same, or is it not possible to tell?
    \end{enumerate} 
}

\begin{align*}
    \intertext{Continuing my notation from parts (b)-(d), in the long run,}
    \lim_{t\to\infty}f'(k) &= f'(k_2^*)
    \intertext{So the change in $r-g$ in the long run will be}
    (r|_{g_2}-g_2) - (r|_{g_1}-g_1) &= (f'(k_2^*)-g_2) - (f'(k_1^*)-g_1) \\
        &= (\rho +\theta g_2 -g_2) - (\rho +\theta g_1-g_1) \\
        &= (\rho +(\theta-1) g_2) - (\rho +(\theta-1)g_1) \\
        &= (\theta-1) (g_2-g_1) \\
        &\begin{cases}
            >0 &\text{ if } \theta\in (0,1) \\
            <0 &\text{ if } \theta > 1
        \end{cases}
\end{align*}

Since there is no upper constraint on $\theta$ from the utility function ($\theta\in (0,1)$ and $\theta > 1$ both produce an upward sloping, convex function), then there is no way to tell, in general, if $r-g$ rises or falls without an estimate of $\theta$.


%%%%%%%%%%%%%%%%%
%     Part f    %
%%%%%%%%%%%%%%%%%
\newpage\problem{}{
    \begin{enumerate}[label=(\alph*)]
    \setcounter{enumi}{5}
    \item Find an expression for the impact of a marginal change in $g$ on the fraction of output that is saved on the balanced growth path. Can one tell whether this expression is positive or negative?
    \end{enumerate} 
}


%%%%%%%%%%%%%%%%%
%     Part g    %
%%%%%%%%%%%%%%%%%
\newpage\problem{}{
    \begin{enumerate}[label=(\alph*)]
    \setcounter{enumi}{6}
    \item For the case where the production function is Cobb-Douglas, $f'(k^*) = k^\alpha$, rewrite your answer to part (f) in terms of $\rho, n, g, \theta,$ and $\alpha$. (Hint: Use the fact that $f'(k^*) = \rho + \theta g$.)
    \end{enumerate} 
}



%%%%%%%%%%%%%%%%%%%%%%%%%%%%%%%%%%%%%%%%%%%%%%%
%                Problem 2                    %
%%%%%%%%%%%%%%%%%%%%%%%%%%%%%%%%%%%%%%%%%%%%%%%
\newpage
\section*{Problem 2}
\problem{Capital taxation in the Ramsey-Cass-Koopmans model: Romer 2.10}{
    Consider a Ramsey-Cass-Koopmans economy that is on its balanced growth path. Suppose
that at some time, which we will call time 0, the government switches to a policy
of taxing investment income at rate $\tau$. Thus the real interest rate that households
face is now given by \[r(t) = (1-\tau) f'(k(t)).\] Assume that the government returns
the revenue it collects from this tax through lump-sum transfers. Finally, assume
that this change in tax policy is unanticipated.

    \begin{enumerate}[label=(\alph*)]
    \item How, if at all, does the tax affect the $\ccdot = 0$ locus? The $\kdot = 0$ locus?
    \end{enumerate} 
}
\def\kone{k^*_1} \def\ktwo{k^*_2}
\begin{align*}
    \intertext{To examine the effects on the $\ccdot=0$ locus, we set $\ccdot=0$ in the consumption Euler equation. This give}
    r(t) &= \rho + \theta g
    \intertext{Before the policy is introduced, we know $f'(k)=r(t)$ because firms are operating under a CRS production function in a competitive market. So let $\kone$ denote $k$ when $\ccdot=0$ before the tax is introduced, thus}
    f'(\kone) &= \rho + \theta g
    \intertext{After the tax is introduced, $r(t)=(1-\tau) f'(k(t))$, so let $\ktwo$ denote $k$ when $\ccdot=0$ after the tax is introduced, satisfying}
    (1-\tau)f'(\ktwo) &= \rho + \theta g\\[1em]
    f'(\ktwo) &= \frac{\rho + \theta g}{1-\tau}
    \intertext{Since $\tau\in(0,1)$, then $(1-\tau)^{-1}\in(1,\infty)$,so }
    \frac{\rho + \theta g}{1-\tau} &> \rho + \theta g \\[1em]
    f'(\ktwo) &> f'(\kone)
    \intertext{Because $f'>0$ and $f''<0$, this implies}
    \ktwo &< \kone
    \intertext{So, the $\ccdot=0$ locus is shifting left from a vertical line (in $k-c$ space) at $\kone$ to a vertical line at $\ktwo$.}
\end{align*} 

\newpage
\begin{align*}
    \intertext{To examine the effects on the $\kdot=0$ locus, we set $\kdot=0$ in the intensive form of the equation of motion for capital. This gives}
    c &= f(k) - (n+g)k
    \intertext{This equation is not related to $r(t)$ and does no change the $\kdot=0$ path through $k-c$ space. The new tax only changes how households trade off consumption for saving and where the $\ccdot=0$ locus intersects the $\kdot=0$ path.}
\end{align*} 

%%%%%%%%%%%%%%%%%
%     Part b    %
%%%%%%%%%%%%%%%%%
\newpage\problem{}{
    \begin{enumerate}[label=(\alph*)]
    \setcounter{enumi}{1}
    \item How does the economy respond to the adoption of the tax at time 0? What are the dynamics after time 0?
    \end{enumerate} 
}
$\kdot$

%%%%%%%%%%%%%%%%%
%     Part c    %
%%%%%%%%%%%%%%%%%
\newpage\problem{}{
    \begin{enumerate}[label=(\alph*)]
    \setcounter{enumi}{2}
    \item How do the values of $c$ and $k$ on the new balanced growth path compare with their values on the old balanced growth path?
    \end{enumerate} 
}


%%%%%%%%%%%%%%%%%
%     Part d    %
%%%%%%%%%%%%%%%%%
\newpage\problem{}{
    \begin{enumerate}[label=(\alph*)]
    \setcounter{enumi}{3}
    \item (This is based on Barro, Mankiw, and Sala-i-Martin, 1995.) Suppose there are many economies like this one. Workers’ preferences are the same in each country, but the tax rates on investment income may vary across countries. Assume that each country is on its balanced growth path.
    
    \begin{enumerate}[label=(i)]
        \item Show that the saving rate on the balanced growth path, $(y^* - c^*)/y^*$, is decreasing in $\tau$.
    \end{enumerate}
    \end{enumerate} 
}



%%%%%%%%%%%%
% Part ii %
%%%%%%%%%%%
\newpage\problem{}{
    \begin{enumerate}[label=(ii)]
    \item Do citizens in low-$\tau$, high-$k^*$, high-saving countries have any incentive to invest in low-saving countries? Why or why not?
    \end{enumerate} 
}



%%%%%%%%%%%%%%%%%
%     Part e    %
%%%%%%%%%%%%%%%%%
\newpage\problem{}{
    \begin{enumerate}[label=(\alph*)]
    \setcounter{enumi}{4}
    \item Does your answer to part (c) imply that a policy of \textit{subsidizing} investment (that is, making $\tau<0$), and raising the revenue for this subsidy through lump-sum taxes, increases welfare? Why or why not?
    \end{enumerate} 
}

\includegraphics[width=0.8\textwidth]{2.b}

%%%%%%%%%%%%%%%%%
%     Part f    %
%%%%%%%%%%%%%%%%%
\newpage\problem{}{
    \begin{enumerate}[label=(\alph*)]
    \setcounter{enumi}{5}
    \item How, if at all, do the answers to parts (a) and (b) change if the government does not rebate the revenue from the tax but instead uses it to make government purchases?
    \end{enumerate} 
}





\end{document}

