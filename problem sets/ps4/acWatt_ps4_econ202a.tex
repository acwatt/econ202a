\documentclass[12pt]{article}
\usepackage{{../preamble}}
\graphicspath{{pics}}

\begin{document}
\chead{Problem Set 4}

%%%%%%%%%%%%%%%%%%%%%%%%%%%%%%%%%%%%%%%%%%%%%%%
%                  Definitions                %
%%%%%%%%%%%%%%%%%%%%%%%%%%%%%%%%%%%%%%%%%%%%%%%
\def\xdot{\dot X} \def\ydot{\dot Y} \def\zdot{\dot Z}
\def\kdot{\dot k} \def\ccdot{\dot c} \def\ddt{\frac{d}{dt}}
\def\dfdk{\frac{\partial F}{\partial K}}
\def\dfdl{\frac{\partial F}{\partial L}}
\def\plim{\lim\limits_{\phi\to 0}}
\def\ddp{\dfrac{\partial}{\partial \phi}}
\def\dgdp{\dfrac{\partial\gamma}{\partial \phi}}
\def\invphi{\sfrac{1}{\phi}}
\def\kone{k^*_1} \def\ktwo{k^*_2}




%%%%%%%%%%%%%%%%%%%%%%%%%%%%%%%%%%%%%%%%%%%%%%%
%                Problem 1                    %
%%%%%%%%%%%%%%%%%%%%%%%%%%%%%%%%%%%%%%%%%%%%%%%
\section*{Problem 1}
\problem{Growth Accounting in Solow: Romer 1.15}{
    Consider a Solow economy on its balanced growth path. Suppose the growth accounting techniques described in Section 1.7 are applied to this economy.

    \begin{enumerate}[label=(\alph*)]
    \item What fraction of growth in output per worker does growth accounting attribute to growth in capital per worker? What fraction does it attribute to technological progress?
    \end{enumerate} 
}
\def\ydoty{\frac{\dot Y}{Y}}
\def\adota{\frac{\dot A}{A}}
\def\kdotk{\frac{\dot K}{K}}
\def\ldotl{\frac{\dot L}{L}}
\begin{align*}
\intertext{From section 1.7, we know that the growth rate of output per worker in the Solow model can be expressed as}
    \ydoty - \ldotl &= \alpha_K \l[\kdotk - \ldotl \r] + R
\intertext{On the balanced growth path, we know that}
    \ydoty &= n+g, \quad
    \ldotl = n, \quad
    \kdotk = n+g
\intertext{and using logs and derivatives, we know that}
    \frac{\dot y}{y} &= \ydoty - \ldotl \implies \frac{\dot y}{y} = g
\intertext{Plugging these balanced growth path properties into the first equation, we have}
    n+g-n &= \alpha_K (n+g-n) + R\\
    g &= \alpha_K g + R\\
    \implies R &= (1-\alpha_K)g
\intertext{So, on the balanced growth path, we know output per worker is growing at rate $g$, and that growth in capital per worker contributes $\alpha_K g$ and technological progress contributes $(1-\alpha_K)g$. So the fraction of growth in output per worker attributable to growth in capital per worker is $\alpha_K$, and $(1-\alpha_K)$ is the fraction attributable to growth in technological progress.}
\end{align*}




%%%%%%%%%%%%%%%%%
%     Part b    %
%%%%%%%%%%%%%%%%%
\newpage\problem{}{
    \begin{enumerate}[label=(\alph*)]
    \setcounter{enumi}{1}
    \item How can you reconcile your results in (a) with the fact that the Solow model implies that the growth rate of output per worker on the balanced growth path is determined solely by the rate of technological progress?
    \end{enumerate} 
}

As Romer states, growth accounting only examines the immediate determinants of growth -- it is not accounting for what is driving the growth of capital versus technological progress. Since we know that long-term growth in the Solow model is only coming from technological progress, we can guess that, in the long run, technological progress is causing capital per worker to increase. So when we look in the short run using growth accounting, we see capital per worker contributing to growth in output per worker, but this is merely growth in technological progress indirectly affecting output via capital per worker.



%%%%%%%%%%%%%%%%%%%%%%%%%%%%%%%%%%%%%%%%%%%%%%%
%                Problem 2                    %
%%%%%%%%%%%%%%%%%%%%%%%%%%%%%%%%%%%%%%%%%%%%%%%
\newpage
\section*{Problem 2}
\problem{Convergence and Measurement Error: Romer 1.16}{
    

    \begin{enumerate}[label=(\alph*)]
    \item In the model of convergence and measurement error in equations (1.39) and (1.40), suppose the true value of $b$ is -1. Does a regression of $\ln(Y/N )_{1979} - \ln(Y/N)_{1870}$ on a constant and $\ln(Y/N)_{1870}$ yield a biased estimate of $b$? Explain.
    \end{enumerate} 
}
\def\tx{\tilde x}
\begin{align*}
\intertext{Let $y=\ln(Y/N )_{1979}$ and $x=\ln(Y/N)_{1870}$, then the linear model of the true data would look like}
    y - x &= a + bx +\varepsilon
\intertext{Assuming that there is some measurement error $u$ in $x$, so we actually measure $\tx = x + u$ the model becomes}
    y - x &= a + bx +\varepsilon \\
    y - (\tx - u) &= a + b(\tx-u) + \varepsilon
\intertext{Rearranging the terms so both the econometric error and measurement error are lumped into the residual, we have}
    y - \tx &= a + b\tx + (\varepsilon - bu - u) \\
        &= a + b\tx + (\varepsilon - (b+1)u)
\intertext{If $b\neq -1$, then we would have an endogeneity bias because $u$ is in the error term and correlated with $\tx$. Because $u$ and $\tx$ are positively correlated, we can sign this bias -- if $b>-1$, the bias is negative. However, if the true value of $b$ is exactly -1, the model reverts to}
    y - x &= a + bx + \varepsilon
\intertext{and there would be no bias in our OLS estimate of $b$. Thus, an OLS estimate of $b$ of -1 is indistinguishable from the true value of $b$ being -1 or a case of measurement error biasing down our OLS estimate.}
\end{align*}







%%%%%%%%%%%%%%%%%
%     Part b    %
%%%%%%%%%%%%%%%%%
\newpage\problem{}{
    \begin{enumerate}[label=(\alph*)]
    \setcounter{enumi}{1}
    \item Suppose there is measurement error in measured 1979 income per capita but not in 1870 income per capita. Does a regression of $\ln(Y/N )_{1979} - \ln(Y/N)_{1870}$ on a constant and $\ln(Y/N)_{1870}$ yield a biased estimate of b? Explain.
    \end{enumerate} 
}

\def\ty{\tilde y}
\begin{align*}
\intertext{Again, let $y=\ln(Y/N )_{1979}$ and $x=\ln(Y/N)_{1870}$, then the linear model of the true data would look like}
    y - x &= a + bx +\varepsilon
\intertext{In the case where $y$ has measurement error $v$, we actually measure $\ty = y + v$ the model becomes}
    y - x &= a + bx +\varepsilon \\
    \ty - v - x &= a + bx + \varepsilon
\intertext{Rearranging the terms so both the econometric error and measurement error are lumped into the residual, we have}
    \ty - x &= a + bx + (\varepsilon +v)
\intertext{If we assume that the measurement error in $y$ is independent of $x$, then there is no endogeneity bias coming from the measurement error $v$. Thus OLS of $\ty$ on $x$ and a constant will give us unbiased estimates of $b$, though the standard errors will be larger than with no measurement error.}
\end{align*}




%%%%%%%%%%%%%%%%%%%%%%%%%%%%%%%%%%%%%%%%%%%%%%%
%                Problem 3                    %
%%%%%%%%%%%%%%%%%%%%%%%%%%%%%%%%%%%%%%%%%%%%%%%
\newpage
\section*{Problem 3}
\problem{Education Consequences in Production Function 1: Romer 4.3}{
    Suppose output in country i is given by $Y_i = A_i Q_i e^{\phi E_i} L_i$. Here $E_i$ is each worker’s years of education, $Q_i$ is the quality of education, and the rest of the notation is standard. Higher output per worker raises the quality of education. Specifically, $Q_i$ is given by $B_i(Y_i/L_i)^\gamma,\ 0 < \gamma < 1,\ B_i > 0$.
    
    Our goal is to decompose the difference in log output per worker between two countries, 1 and 2, into the contributions of education and all other forces. We have data on $Y$, $L$, and $E$ in the two countries, and we know the values of the parameters $\phi$ and $\gamma$.

    \begin{enumerate}[label=(\alph*)]
        \item Explain in what way attributing amount $\phi(E_2 - E_1)$ of $\ln(Y_2/L_2) - \ln(Y_1/L_1)$ to education and the remainder to other forces would understate the contribution of education to the difference in log output per worker between the two countries.
    \end{enumerate} 
}

\begin{align*}
\intertext{For the moment, I will drop the $i$ subscripts for ease of exposition. Plugging in our form of $Q$ into $Y$, we have}
    Y &=  AB\l(\frac{Y}{L}\r)^\gamma e^{\phi E}L
\intertext{taking logs we have}
    \ln Y &= \ln A B +\gamma (\ln Y - \ln L) + \phi E + \ln L
\intertext{Solving for $\ln Y - \ln L$ we get}
    \ln \frac{Y}{L} &= \frac{1}{1-\gamma}\ln A B + \frac{\phi}{1-\gamma}E
\intertext{So the difference in log output per worker between country 1 and 2 is}
\ln \frac{Y_2}{L_2} - \ln\frac{Y_1}{L_1} &= \frac{1}{1-\gamma}\ln \frac{A_2B_2}{A_1B_1} + \frac{\phi}{1-\gamma}(E_2-E_1)
\intertext{Since $\gamma \in (0,1)$, then $\dfrac{1}{1-\gamma}>1$ and }
    \phi(E_2-E_1) &< \frac{\phi}{1-\gamma}(E_2-E_1)
\intertext{So $\phi(E_2-E_1)$ would understate the fraction of the difference in log output per worker by a factor of $\dfrac{1}{1-\gamma}$.}
\end{align*}




%%%%%%%%%%%%%%%%%
%     Part b    %
%%%%%%%%%%%%%%%%%
\newpage\problem{}{
    \begin{enumerate}[label=(\alph*)]
    \setcounter{enumi}{1}
    \item What would be a better measure of the contribution of education to the difference in log output per worker?
    \end{enumerate} 
}

\begin{align*}
\intertext{Assuming technological progress is independent of education, then }
\frac{\phi}{1-\gamma}(E_2-E_1)&
\intertext{would be a better measure of the contribution of education to the difference in log output per worker. However, we know that education has major impacts on innovation and technological development, so we would want to account for how $E$ is changing the path of $A$ in the long run. }
\end{align*}







%%%%%%%%%%%%%%%%%%%%%%%%%%%%%%%%%%%%%%%%%%%%%%%
%                Problem 4                    %
%%%%%%%%%%%%%%%%%%%%%%%%%%%%%%%%%%%%%%%%%%%%%%%
\newpage
\section*{Problem 4}
\problem{Education Consequences in Production Function 2: Romer 4.4}{
    Suppose the production function is $Y = K^\alpha (e^{\phi E} L)^{1-\alpha}, 0<\alpha<1$. $E$ is the amount of education workers receive; the rest of the notation is standard. Assume that there is perfect capital mobility. In particular, $K$ always adjusts so that the marginal product of capital equals the world rate of return, $r^*$.

    \begin{enumerate}[label=(\alph*)]
        \item Find an expression for the marginal product of capital as a function of $K$, $E$, $L$, and the parameters of the production function.
    \end{enumerate} 
}

Ran out of time to finish, oops!


%%%%%%%%%%%%%%%%%
%     Part b    %
%%%%%%%%%%%%%%%%%
\newpage\problem{}{
    \begin{enumerate}[label=(\alph*)]
    \setcounter{enumi}{1}
    \item Use the equation you derived in (a) to find $K$ as a function of $r^*$, $E$, $L$, and the parameters of the production function.
    \end{enumerate} 
}


%%%%%%%%%%%%%%%%%
%     Part c    %
 %%%%%%%%%%%%%%%%% \newpage
\newpage
\problem{}{
    \begin{enumerate}[label=(\alph*)]
    \setcounter{enumi}{2}
    \item Use your answer in (b) to find an expression for $d(\ln Y)/dE$, incorporating the effect of $E$ on $Y$ via $K$.
    \end{enumerate} 
}


%%%%%%%%%%%%%%%%%
%     Part d    %
%%%%%%%%%%%%%%%%%
\newpage\problem{}{
    \begin{enumerate}[label=(\alph*)]
    \setcounter{enumi}{3}
    \item Explain intuitively how capital mobility affects the impact of the change in $E$ on output.
    \end{enumerate} 
}



\end{document}

