\documentclass[12pt]{article}
\usepackage{{../preamble}} % for use when the .tex file is in a psX subfolder
\graphicspath{{pics}}
\begin{document}
% \maketitle
\chead{Problem Set 5}

%%%%%%%%%%%%%%%%%%%%%%%%%%%%%%%%%%%%%%%%%%%%%%%
%                  Definitions                %
%%%%%%%%%%%%%%%%%%%%%%%%%%%%%%%%%%%%%%%%%%%%%%%
%\includegraphics[width=\mywidth\textwidth]{}

% \begin{figure}[h!]
% \centering
% \input{pics/PS2/p7b}
% \caption{}
% \label{fig-}
% \end{figure}

% \begin{enumerate}[label=\alph*.]
%     \setcounter{enumi}{1}
%     \item 
% \end{enumerate}

%%%%%%%%%%%%%%%%%
%     Part a    %
%%%%%%%%%%%%%%%%%


\def\xdot{\dot X} \def\ydot{\dot Y} \def\zdot{\dot Z}
\def\kdot{\dot k} \def\ccdot{\dot c} \def\ddt{\frac{d}{dt}}
\def\dfdk{\frac{\partial F}{\partial K}}
\def\dfdl{\frac{\partial F}{\partial L}}
\def\plim{\lim\limits_{\phi\to 0}}
\def\ddp{\dfrac{\partial}{\partial \phi}}
\def\dgdp{\dfrac{\partial\gamma}{\partial \phi}}
\def\invphi{\sfrac{1}{\phi}}
\def\kone{k^*_1} \def\ktwo{k^*_2}
\def\adot{\dot A}
\def\ga{g_A^*}
\def\gk{g_K^*}
\def\gadot{\dot g_A} \def\gkdot{\dot g_K}
\newcommand{\growth}[1]{\frac{\dot #1}{#1}}


%%%%%%%%%%%%%%%%%%%%%%%%%%%%%%%%%%%%%%%%%%%%%%%
%                Problem 1                    %
%%%%%%%%%%%%%%%%%%%%%%%%%%%%%%%%%%%%%%%%%%%%%%%
\section*{Problem 1}
\problem{No-Capital Model: Romer 3.1}{
    Consider the model of Section 3.2 with $\theta<1$.

    \begin{enumerate}[label=(\alph*)]
    \item On the balanced growth path, $\adot = g_A^* A$ where $g_A^*$ is the balanced-growth-path value of $g_A$. Use this fact and equation (3.6) to derive an expression for $A(t)$ on the balanced growth path in terms of $B, a_L , \gamma, \theta,$ and $L(t )$.
    \end{enumerate} 
}

\begin{align*} 
\intertext{First, we can begin by finding an expression for $g_A$.}
    \adot &= \ga A \\
    \adot &= B(a_L L)^\gamma A^\theta \\
    \implies \ga &=  B(a_L L)^\gamma A^{\theta-1}
\intertext{On the balanced growth path, the growth rate $g_A$ is constant. So $\dot g_A=0$. We can find $\dot g_A$ by taking the time derivative of $\ln\ga$}
    \ln\ga &= \ln(Ba_L^\gamma) + \gamma \ln L + (\theta-1)\ln A \\
    \growth{\ga} &= \gamma \growth{L} + (\theta-1)\growth{A} \\
        &= \gamma n + (\theta-1)g_A
    \intertext{Since $\dot g_A=0$ on the BGP,}
    \ga &= \frac{\gamma n}{(1-\theta)}
\intertext{From earlier, this implies}
    \frac{\gamma n}{(1-\theta)} &= B(a_L L)^\gamma A^{\theta-1}
\intertext{And solving for $A$ yields}
    A &= \l[B(a_L L)^\gamma\  \frac{(1-\theta)}{\gamma n}\r]^{\dfrac{1}{1-\theta}}
\end{align*}





%%%%%%%%%%%%%%%%%
%     Part b    %
%%%%%%%%%%%%%%%%%
\newpage\problem{}{
    \begin{enumerate}[label=(\alph*)]
    \setcounter{enumi}{1}
    \item Use your answer to part (a) and the production function, (3.5), to obtain an expression for $Y(t)$ on the balanced growth path. Find the value of $a_L$ that maximizes output on the balanced growth path.
    \end{enumerate} 
}


\begin{align*}
    Y &= A(1-a_L)L
\intertext{Plugging in our $A$ function from part (a), we get}
    Y &= \l[B(a_L L)^\gamma\  \frac{(1-\theta)}{\gamma n}\r]^{\dfrac{1}{1-\theta}}(1-a_L)L
\intertext{Since our goal is to maximize $Y$, we can simplify by maximizing $\ln Y$ instead:}
    \ln Y &= \frac{1}{(1-\theta)}\ln\l(BL^\gamma\frac{(1-\theta)}{\gamma n} \r) +
        \frac{\gamma}{(1-\theta)}\ln a_L +
        \ln(1-a_L) +\ln L
\intertext{Then the FOC of $\ln Y$ is}
    \part{\ln Y}{a_L} &= \frac{\gamma}{(1-\theta)}\frac{1}{a_L} - \frac{1}{1-a_L}
        =0\\
    \implies a_L &  = \frac{\gamma}{1-\theta +\gamma}
\end{align*}






%%%%%%%%%%%%%%%%%%%%%%%%%%%%%%%%%%%%%%%%%%%%%%%
%                Problem 2                    %
%%%%%%%%%%%%%%%%%%%%%%%%%%%%%%%%%%%%%%%%%%%%%%%
\newpage
\section*{Problem 2}
\problem{A model with Capital: Romer 3.3}{
    Consider the economy analyzed in Section 3.3. Assume that $\theta + \beta <1$ and $n > 0$, and that the economy is on its balanced growth path. Describe how each of the following changes affect the $\dot g_A=0$ and $\dot g_K =0$ lines and the position of the economy in $(g_A, g_K)$ space at the moment of the change:

    \begin{enumerate}[label=(\alph*)]
    \item An increase in $n$.
    \end{enumerate} 
}
\begin{align*} 
\intertext{The $\gadot=0$ line is defined by}
    \gadot &= 0 =\beta g_K +\gamma n +(\theta-1)g_A\\
    \implies g_K &= \frac{1-\theta}{\beta}g_A - \frac{\gamma}{\beta}n
\intertext{Which means the $\gadot=0$ line shifts vertically downward by $\dfrac{\gamma}{\beta}$ for every 1-unit increase in $n$.}
\intertext{Similarly, the $\gkdot=0$ line is defined by}
    \gkdot &= 0 = (1-\alpha)(g_A+n-g_K)\\
    \implies g_K &= g_A + n
\intertext{So the $\gkdot=0$ line shifts vertically up by the amount $n$ has increased.}
\intertext{In the case that $\theta + \beta <1$, we know that }
    \ga &= \frac{\beta+\gamma}{1-(\theta+\beta)}n
\intertext{So $\ga$ has shifted out by $\dfrac{\beta+\gamma}{1-(\theta+\beta)}$ for every 1-unit increase in $n$.}
\intertext{We know that $g_K = g_A + n$, so}
    \gk &= \l[\frac{\beta+\gamma}{1-(\theta+\beta)} + 1\r]n
\intertext{And $\gk$ shifts up. Thus the new BGP position of the economy in $(g_A, g_K)$ space at the moment of the change is out and up compared to the old position in $(g_A, g_K)$ space.}
\intertext{From section 3.3, we know that}
    g_K &= s(1-a_K)^\alpha (1-a_L)^{1-\alpha}
        \l[ \frac{AL}{K}\r]^{1-\alpha}
\intertext{and}
    g_A &= B a_K^\beta a_L^\gamma 
        K^\beta L^\gamma A^{\theta-1}
\intertext{Since $n$ isn't in either of these equations, and $K,L,A$ all need to change continuously, there is no jump in the location of the economy in $(g_A, g_K)$ space at the moment of the change. The economy smoothly moves outward in $(g_A, g_K)$ space toward the new $\ga,\gk$ location.}
\end{align*}





%%%%%%%%%%%%%%%%%
%     Part b    %
%%%%%%%%%%%%%%%%%
\newpage\problem{}{
    \begin{enumerate}[label=(\alph*)]
    \setcounter{enumi}{1}
    \item An increase in $a_K$.
    \end{enumerate} 
}

\begin{align*}
\intertext{From part (a), we have that }
    \ga &= \frac{\beta+\gamma}{1-(\theta+\beta)}n\\
    \gk &= \l[\frac{\beta+\gamma}{1-(\theta+\beta)} + 1\r]n
\intertext{Thus there is no change in the BGP location in $(g_A, g_K)$ space. }
\intertext{From section 3.3, we know that}
    g_K &= s(1-a_K)^\alpha (1-a_L)^{1-\alpha}
        \l[ \frac{AL}{K}\r]^{1-\alpha}
\intertext{and}
    g_A &= B a_K^\beta a_L^\gamma 
        K^\beta L^\gamma A^{\theta-1}
\intertext{So immediately after the increase in $a_K$, the economy jumps to a point down and to the right of the old BGP location. And then the economy will decrease $g_A$ and increase $g_K$ in order to get back to the BGP location.}
\end{align*}




%%%%%%%%%%%%%%%%%
%     Part c    %
%%%%%%%%%%%%%%%%%
\newpage\problem{}{
    \begin{enumerate}[label=(\alph*)]
    \setcounter{enumi}{2}
    \item An increase in $\theta$.
    \end{enumerate} 
}

\begin{align*}
\intertext{From section 3.3, we know that}
    g_K &= s(1-a_K)^\alpha (1-a_L)^{1-\alpha}
        \l[ \frac{AL}{K}\r]^{1-\alpha}\\
    g_A &= B a_K^\beta a_L^\gamma 
        K^\beta L^\gamma A^{\theta-1}
\intertext{So the economy does not jump in $(g_A, g_K)$ space. 
We also know that }
    \ga &= \frac{\beta+\gamma}{1-(\theta+\beta)}n\\
    \gk &= \l[\frac{\beta+\gamma}{1-(\theta+\beta)} + 1\r]n
\intertext{So $\ga$ shifts to inward (to the left) and $\gk$ shifts downward. Thus the economy moves smoothly down and to the left in $(g_A, g_K)$ space.}
\end{align*}


%%%%%%%%%%%%%%%%%%%%%%%%%%%%%%%%%%%%%%%%%%%%%%%
%                Problem 3                    %
%%%%%%%%%%%%%%%%%%%%%%%%%%%%%%%%%%%%%%%%%%%%%%%
\newpage
\section*{Problem 3}
\problem{Identical capital and knowledge growth rates: Romer 3.5}{
    Consider the model of Section 3.3 with $\beta + \theta = 1$ and $n = 0$.

    \begin{enumerate}[label=(\alph*)]
    \item Using (3.14) and (3.16), find the value that $A/K$ must have for $g_K$ and $g_A$ to be equal.
    \end{enumerate} 
}

\begin{align*} 
\intertext{Equations (3.14) and (3.16) give}
    g_K &= s(1-a_K)^\alpha (1-a_L)^{1-\alpha}
        \l[ \frac{AL}{K}\r]^{1-\alpha}\\
    g_A &= B a_K^\beta a_L^\gamma 
        K^\beta L^\gamma A^{\theta-1}
\intertext{Setting these equal gives}
    g_K &= g_A \\
    s(1-a_K)^\alpha (1-a_L)^{1-\alpha}
        \l( \frac{AL}{K}\r)^{1-\alpha}
    &=B a_K^\beta a_L^\gamma 
        K^\beta L^\gamma A^{\theta-1}\\
\intertext{Since $\beta + \theta = 1$, the right hand side becomes}
    s(1-a_K)^\alpha (1-a_L)^{1-\alpha}
        \l( \frac{AL}{K}\r)^{1-\alpha}
    &=B a_K^\beta a_L^\gamma 
        \l(\frac{A}{K}\r)^{-\beta} L^\gamma\\
    s(1-a_K)^\alpha (1-a_L)^{1-\alpha}
        \l(\frac{A}{K}\r)^{1-\alpha}L^{1-\alpha}
    &=B a_K^\beta a_L^\gamma 
        \l(\frac{A}{K}\r)^{-\beta} L^\gamma\\
    \l(\frac{A}{K}\r)^{1-\alpha+\beta}
    &=\frac{B a_K^\beta a_L^\gamma 
         L^{\gamma+\alpha-1}}{s(1-a_K)^\alpha (1-a_L)^{1-\alpha}}\\
    \frac{A}{K}
    &=\l[\frac{B a_K^\beta a_L^\gamma 
         L^{\gamma+\alpha-1}}{s(1-a_K)^\alpha (1-a_L)^{1-\alpha}}\r]^{\dfrac{1}{1-\alpha+\beta}}\\
\end{align*}





%%%%%%%%%%%%%%%%%
%     Part b    %
%%%%%%%%%%%%%%%%%
\newpage\problem{}{
    \begin{enumerate}[label=(\alph*)]
    \setcounter{enumi}{1}
    \item Using your result in part (a), find the growth rate of $A$ and $K$ when $g_K = g_A$.
    \end{enumerate} 
}

\begin{align*}
\intertext{Plugging in $A/K$ from part (a) into equation (3.14), we get}
    g \equiv g_K &= s(1-a_K)^\alpha (1-a_L)^{1-\alpha}
        \l[ \frac{AL}{K}\r]^{1-\alpha}\\
    &= s(1-a_K)^\alpha (1-a_L)^{1-\alpha}
        \l[\frac{B a_K^\beta a_L^\gamma 
         L^{\gamma+\alpha-1}}{s(1-a_K)^\alpha (1-a_L)^{1-\alpha}}\r]^{\dfrac{1-\alpha}{1-\alpha+\beta}}
         L^{1-\alpha}\\
\intertext{And consolidating the $L$ terms, we have}
    &= s(1-a_K)^\alpha (1-a_L)^{1-\alpha}
        \l[\frac{B a_K^\beta a_L^\gamma}{s(1-a_K)^\alpha (1-a_L)^{1-\alpha}}\r]^{\dfrac{1-\alpha}{1-\alpha+\beta}}
         L^{\frac{(\gamma+\alpha-1)(1-\alpha)}{1-\alpha+\beta}}L^{1-\alpha}\\
    &= s(1-a_K)^\alpha (1-a_L)^{1-\alpha}
        \l[\frac{B a_K^\beta a_L^\gamma}{s(1-a_K)^\alpha (1-a_L)^{1-\alpha}}\r]^{\dfrac{1-\alpha}{1-\alpha+\beta}}
         L^{\dfrac{(\gamma+\beta)(1-\alpha)}{1-\alpha+\beta}}\\
\intertext{And simplifying the $s,a_K,a_L$ terms, we have}
    &= \l[s(1-a_K)^\alpha (1-a_L)^{1-\alpha} \r]^{\dfrac{\beta}{1-\alpha+\beta}}
        \l[B a_K^\beta a_L^\gamma\r]^{\dfrac{1-\alpha}{1-\alpha+\beta}}
        L^{\dfrac{(\gamma+\beta)(1-\alpha)}{1-\alpha+\beta}}
\end{align*}




%%%%%%%%%%%%%%%%%
%     Part c    %
%%%%%%%%%%%%%%%%%
\newpage\problem{}{
    \begin{enumerate}[label=(\alph*)]
    \setcounter{enumi}{2}
    \item How does an increase in $s$ affect the long-run growth rate of the economy?
    \end{enumerate} 
}



\begin{align*}
\intertext{Since $n=0$, the growth rate of the economy is }
    \frac{\ydot}{Y} &= \alpha g_K + (1-\alpha)g_A
\intertext{and since $g_K = g_A = g$}
    &= g\\
    &= \l[s(1-a_K)^\alpha (1-a_L)^{1-\alpha} \r]^{\dfrac{\beta}{1-\alpha+\beta}}
        \l[B a_K^\beta a_L^\gamma\r]^{\dfrac{1-\alpha}{1-\alpha+\beta}}
        L^{\dfrac{(\gamma+\beta)(1-\alpha)}{1-\alpha+\beta}}
\intertext{So an increase in $s$ by 1\% will increase the long-run growth rate of the economy by approximately a factor of $1.01^{\dfrac{\beta}{1-\alpha+\beta}}$. Since $\frac{\beta}{1-\alpha+\beta}<1$, this factor is between 1 and 1.01. }
\end{align*}




%%%%%%%%%%%%%%%%%
%     Part d    %
%%%%%%%%%%%%%%%%%
\newpage\problem{}{
    \begin{enumerate}[label=(\alph*)]
    \setcounter{enumi}{3}
    \item What value of $a_K$ maximizes the long-run growth rate of the economy? Intuitively, why is this value not increasing in $\theta$, the importance of capital in the R\&D sector?
    \end{enumerate} 
}



\begin{align*}
\intertext{To maximize the long-run growth, we want to maximize $g$. We can also maximize $\ln g$:}
    \ln g &= \frac{\beta}{1-\alpha+\beta}\ln\l[s(1-a_K)^\alpha (1-a_L)^{1-\alpha} \r] + 
        \frac{1-\alpha}{1-\alpha+\beta}\ln\l[B a_K^\beta a_L^\gamma\r] +
        \frac{(\gamma+\beta)(1-\alpha)}{1-\alpha+\beta}\ln L \\
    &= \frac{\beta\alpha}{1-\alpha+\beta}\ln(1-a_K) + 
        \frac{(1-\alpha)\beta}{1-\alpha+\beta}\ln a_K + \text{non-$a_K$ terms}
\intertext{Then the FOC for $a_K$ is}
    \frac{(1-\alpha)\beta}{(1-\alpha+\beta)a_K} 
        &= \frac{\beta\alpha}{(1-\alpha+\beta)(1-a_K)}\\
    a_K\alpha &= (1-a_K)(1-\alpha)\\
        &= 1 -\alpha - a_K(1-\alpha)\\
    (\alpha +1 -\alpha)a_K &= 1-\alpha \\
    a_K &= 1-\alpha
\end{align*}


In this case of the model, $\beta = 1-\theta$. So, if $\theta$ increased, the effectiveness of knowledge would increase in the sense that an additional unit of $A$ would have larger impacts on $g_A$. But an increase in $\theta$ $\implies$ an decrease in $\beta$, which would decrease the effectiveness of $K$ on $g_A$. So in this case of the model, we can think of the effects of $\theta$ as cancelling out with the effects of $\beta$ on the growth-rate maximizing value of $a_K$.

\end{document}

